%%!TEX TS-program = xelatexmk
\documentclass[oneside,justified,marginals=raggedouter]{tufte-handout}
\usepackage{fontspec,xltxtra,xunicode}
\usepackage{enumerate}
\usepackage{booktabs}
\usepackage[fleqn]{amsmath}
\usepackage{amssymb}
\usepackage{tikz}
\usetikzlibrary{plotmarks}

\defaultfontfeatures{Mapping=tex-text}
\setmainfont[%Mapping=tex-text,
  BoldFont={Haarlemmer MT Std Bold},
  SlantedFont={Haarlemmer MT Std Italic},
  ItalicFont={Haarlemmer MT Std Italic},
  BoldItalicFont={Haarlemmer MT Std Bold Italic},
  SmallCapsFont={Haarlemmer MT Std: +smcp}
]{Haarlemmer MT Std}
\setsansfont[Scale=MatchLowercase,Mapping=tex-text]{TeX Gyre Heros}
\setmonofont[Scale=MatchLowercase,Ligatures={NoRequired,NoCommon,NoContextual}]{TeX Gyre Cursor}

\DeclareMathOperator*{\argmax}{arg\,max}

% XeTeX workaround for tufte-latex (otherwise need to use xetex or nols options)
\renewcommand{\allcapsspacing}[1]{{\addfontfeature{LetterSpace=20.0}#1}}
\renewcommand{\smallcapsspacing}[1]{{\addfontfeature{LetterSpace=5.0}#1}}
\renewcommand{\textsc}[1]{\smallcapsspacing{\textsmallcaps{#1}}}
\renewcommand{\smallcaps}[1]{\smallcapsspacing{\scshape\MakeTextLowercase{#1}}}

\widowpenalty=1000
\clubpenalty=1000
\raggedbottom
\hyphenpenalty=5000
\tolerance=1000

\title{LING 572 Homework 8}
\author{David McHugh and Chris Curtis}
\date{16 Mar 2013}

\clearpage\relax

\begin{document}
\maketitle


\section{Question 3}

Results on the binary classification task:
\vskip\baselineskip

\begin{tabular}{@{}rrr@{}}
\toprule
N   & Training accuracy & Test accuracy \\ \midrule
1   & 0.452963 & 0.416667 \\
5   & 0.614444 & 0.613333 \\
10  & 0.684444 & 0.710000 \\
20  & 0.753704 & 0.733333 \\
50  & 0.837037 & 0.773333 \\
100 & 0.897407 & 0.753333 \\
150 & 0.926667 & 0.776667 \\
200 & 0.945556 & 0.776667 \\
250 & 0.962963 & 0.763333 \\ \bottomrule
\bottomrule
\end{tabular}
\vskip\baselineskip
From these results, we can see a rapid rise in both training and test
accuracy with modest increases in number of transformations. However,
while training accuracy continues to rise rapidly, test accuracy plateaus
and begins to decline slightly.

This shows that the transformations are
overfitting the training data. With transformation-based learning, it is
conceivable that we could eventually train a model that perfectly fits the
training set—essentially, a 2700-leaf-node decision tree. As these results
show, that tree would generalize poorly to test data, as it is unlikely
a test instance would precisely match any of those leaf nodes.

\end{document}
